% Указываем VS Code использовать XeLaTeX
% !TEX program = xelatex
\PassOptionsToPackage{pdfpagelabels=false}{hyperref}

% Шрифт всего документа - 12pt. Тип документа - статья
\documentclass[12pt]{article}

% Параметры документа (дисциплина, тема, преподаватель и т.п.)
% Загружаем первым, чтобы флаги из config.tex были доступны в преамбуле
% ==========================================
% ПАРАМЕТРЫ ДОКУМЕНТА
% ==========================================

% Библиография:
\newif\ifbib \bibfalse

% Графики (pgfplots):
\newif\ifplots \plotsfalse

% Листинг кода (listings + xcolor):
\newif\iflisting \listingfalse

% Кафедра (две строки)
\newcommand{\kafedraFirstString}{...}
\newcommand{\kafedraSecondString}{...}

% Дисциплина
\newcommand{\discipline}{...}

% Вид работы: «лабораторной» или «семинарской»
\newcommand{\workType}{...}

% Данные студента
\newcommand{\courseYear}{3}
\newcommand{\levelEducation}{бакалавриата}
\newcommand{\group}{ИДБ-23-14}
\newcommand{\lastName}{Дамакальщикова}
\newcommand{\firstName}{Михаила}
\newcommand{\middleName}{Александровича}

% Тема работы
\newcommand{\topic}{...}

% Направление и профиль подготовки
\newcommand{\specialization}{09.03.03 Прикладная информатика}
\newcommand{\trainingProfile}{Математическое и компьютерное моделирование процессов и систем}

% Дата сдачи
\newcommand{\dateSubmission}{}
\newcommand{\monthSubmission}{}
\newcommand{\yearSubmission}{}

% Преподаватель
\newcommand{\teacher}{...}

% Год в колонтитуле
\newcommand{\yearFooter}{...}

% ==========================================
% СОБСТВЕННЫЕ КОМАНДЫ
% ==========================================

% Секция без нумерации
\newcommand{\unnsection}[1]{
    \section*{#1}
    \addcontentsline{toc}{section}{#1}
}

% Подсекция без нумерации
\newcommand{\unnsubsection}[1]{
    \subsection*{#1}
    \addcontentsline{toc}{subsection}{#1}
}


% ==========================================
% ШРИФТЫ И ЯЗЫКИ
% ==========================================

% Управление шрифтами
\usepackage{fontspec}
% Основной шрифт документа
\setmainfont{Liberation Serif}
% \setmainfont{Times New Roman}  % для Windows/macOS
% \setmainfont{Liberation Serif} % для Linux

% Моноширинный шрифт (для листингов кода)
\setmonofont{DejaVu Sans Mono}[Scale=MatchLowercase]
% \setmonofont{Courier New}      % для Windows/macOS
% \setmonofont{Liberation Mono}  % для Linux

% Поддержка многоязычных документов (русский + английский)
\usepackage{polyglossia}
\setdefaultlanguage{russian}
\setotherlanguage{english}

% ==========================================
% МАТЕМАТИКА
% ==========================================

% Расширенные математические формулы и окружения
\usepackage{amsmath}

% Запятая как разделитель дробной части (вместо точки)
\usepackage{icomma}

% ==========================================
% ГРАФИКА И ЦВЕТА
% ==========================================

% Вставка и масштабирование изображений
\usepackage{graphicx}
% Путь к изображениям
\graphicspath{{./images/}}

% Подрисунки (несколько изображений в одном окружении figure)
\usepackage{subcaption}

% Поддержка цветов
\usepackage{xcolor}

% ==========================================
% РАЗМЕТКА СТРАНИЦЫ
% ==========================================

% Настройка полей документа
\usepackage[a4paper, margin=2cm]{geometry}

% ==========================================
% PDF И ССЫЛКИ
% ==========================================

% Кликабельные ссылки в PDF
\usepackage[colorlinks=true, urlcolor=blue, linkcolor=black, citecolor=black]{hyperref}

% Улучшенное управление закладками PDF
\usepackage{bookmark}

% ==========================================
% СОДЕРЖАНИЕ
% ==========================================

% Настройка отображения содержания (точки между заголовком и номером страницы)
\usepackage{tocloft}
\renewcommand{\cftdot}{.}
\renewcommand{\cftsecdotsep}{\cftdotsep}

% ==========================================
% ОПЦИОНАЛЬНЫЕ ПАКЕТЫ
% Управляются флагами в config.tex
% ==========================================

% Библиография в стиле ГОСТ (флаг \bibtrue)
\ifbib
    \usepackage[backend=biber, style=gost-numeric, language=autobib, autolang=other]{biblatex}
    \addbibresource{references.bib}
\fi

% Графики и диаграммы (флаг \plotstrue)
\ifplots
    \usepackage{pgfplots}
    \pgfplotsset{compat=1.18}
    \pgfplotsset{/pgf/number format/use comma}
\fi

% Листинги кода с подсветкой синтаксиса (флаг \listingtrue)
\iflisting
    \usepackage{minted}
    \setminted{
        fontsize=\small,
        linenos=true,
        breaklines=true,
        frame=single
    }
\fi

% Начало документа
\begin{document}

    % Отключает нумерацию страниц в титульном листе и в содержании
    \pagenumbering{gobble}

    % Титульная страница
    % Титульная страница
\begin{titlepage}

% Логотип вуза
\begin{figure}
    \centering
    \includegraphics[width=0.35\linewidth]{logo.png}
\end{figure}

% Подпись под логотипом
\begin{center}
    \textbf{МИНОБРНАУКИ РОССИИ} \\[2pt]
    \textbf{Федеральное государственное автономное образовательное учреждение} \\
    \textbf{высшего образования} \\
    \textbf{«Московский государственный технологический университет «СТАНКИН»} \\
    \textbf{(ФГАОУ ВО «МГТУ «СТАНКИН»)} \\[4pt]
    % Разделительная линия
    \hrule
\end{center}

% Институт
\noindent
\begin{minipage}[t]{0.5\textwidth}
    \raggedright
    \textbf{Институт} \\
    \textbf{информационных} \\
    \textbf{технологий}
\end{minipage}
% Кафедра
\hfill
\begin{minipage}[t]{0.5\textwidth}
    \raggedleft
    \textbf{Кафедра} \\
    \textbf{\kafedraFirstString} \\
    \textbf{\kafedraSecondString}
\end{minipage}
\vspace{1cm}

% Название работы
\begin{center}
    Отчёт о выполнении \workType~работы по дисциплине \\[2pt] «\discipline»
\end{center}
\vspace{0.5cm}
% Кто выполнил работу
\begin{center}
    Студента
    \hspace{0.2ex}
    \underline{\hspace{1ex}\textit{\courseYear}\hspace{1ex}}
    \hspace{0.2ex}
    курса
    \begin{tabular}[t]{c}
        \underline{\hspace{3ex}\textit{\levelEducation}\hspace{3ex}}
        \\[-4pt]
        {\tiny\textit{(уровень профессионального образования)}}
    \end{tabular}
    группы
    \hspace{0.2ex}
    \underline{\hspace{1ex}\textit{\group}\hspace{1ex}} \\[10pt]
    % ФИО
    \underline{\makebox[12cm]{\lastName~\firstName~\middleName}}
\end{center}
\vspace{0.5cm}
% Тема работы
\begin{center}
    На тему \\[3pt] \underline{\makebox[12cm]{\topic}}
\end{center}
\vspace{2cm}

% Направление и профиль подготовки
\noindent
\begin{minipage}[t]{0.5\textwidth}
    \raggedright
    Направление: \\[5pt]
    Профиль подготовки:
\end{minipage}
\begin{minipage}[t]{0.5\textwidth}
    \raggedleft
    \specialization \\[5pt]
    \trainingProfile
\end{minipage}
\vspace{1cm}

% Дата сдачи отчёта
\noindent
Отчёт сдан~~~«\underline{\makebox[4em]{\vphantom{\dateSubmission}}}»~~~\underline{\makebox[10em]{\vphantom{\monthSubmission}}}~~~20\underline{\makebox[2em]{\vphantom{\yearSubmission}}} г.

\vspace{0.5cm}

\noindent
% Оценка
Оценка:~~~\underline{\makebox[3cm]{\vphantom{text}}}

\vspace{0.5cm}

% Преподаватель
\noindent
\begin{minipage}[t]{0.17\textwidth}
    Преподаватель
\end{minipage}
\begin{minipage}[t]{0.7\textwidth}
    \centering
    \underline{\makebox[12cm]{\teacher}} \\
    {\small\textit{(Ф.И.О., должность, степень, звание.)}}
\end{minipage}
\hfill
\begin{minipage}[t]{0.1\textwidth}
    \centering
    \underline{\makebox[2cm]{\vphantom{\teacher}}} \\
    {\small\textit{(подпись)}}
\end{minipage}
\vfil

% Колонтитул
\enlargethispage{3cm}
\centering
\small Москва, \yearFooter~г.

\end{titlepage}


    % Содержание
    \begin{center}
        \tableofcontents
    \end{center}

    % Новая страница
    \newpage

    % Нумерация с 3-его номера
    \pagenumbering{arabic}
    \setcounter{page}{3}

    % ==========================================
    % СОДЕРЖАНИЕ ОТЧЁТА
    % ==========================================

    \unnsection{Начало отчёта}

    \unnsection{Раздел в главе №1}

    Какой-то текст.

    \unnsubsection{Подраздел в главе №1}

        Тоже какой-то текст.


    % Список литературы (управляется флагом \ifbib в config.tex)
    \ifbib
        % Страница со списком литературы
\cleardoublepage

% Делаем так, чтобы библиография появлялась независимо от ссылок
\nocite{*}

% Выводим библиографию по центру
\begin{center}
    \printbibliography[heading=bibintoc, title={Список литературы}]
\end{center}

% Отключаем нумерацию страницы
\thispagestyle{empty}

\cleardoublepage

    \fi

% Конец документа
\end{document}
