% Указываем VS Code использовать XeLaTeX
% !TEX program = xelatex
\PassOptionsToPackage{pdfpagelabels=false}{hyperref}

% Шрифт всего документа - 12pt. Тип документа - статья
\documentclass[12pt]{article}

% Пакет для управления шрифтами
\usepackage{fontspec} 
% Шрифт по умолчанию
\setmainfont{Liberation Serif}
% \setmainfont{Times New Roman}  % для Windows/macOS
% \setmainfont{Liberation Serif} % для Linux

% Шрифт для листинга (раскомментируй при необходимости)
\setmonofont{DejaVu Sans Mono}[Scale=MatchLowercase]
% Альтернативы:
% \setmonofont{Courier New}      % для Windows/macOS
% \setmonofont{Liberation Mono}  % для Linux

% Пакет для работы с языками
\usepackage{polyglossia}
% Язык документа по умолчанию - русский
\setdefaultlanguage{russian}
% Дополнительный язык
\setotherlanguage{english}

% Пакет для использования запятой в качестве разделителя в дробных числах
\usepackage{icomma}

% Пакет для регулирования полей
\usepackage[a4paper, margin=2cm]{geometry}

% Пакет для вставки изображений
\usepackage{graphicx}
% Указываем путь к изображениям (удобно для организации проекта)
\graphicspath{{./images/}}

% Пакет для ссылок в документе
\usepackage{hyperref}

% Улучшения для оглавлений/закладок PDF: загружать после hyperref
\usepackage{bookmark}

% Пакет для разделительных символов в содержании
\usepackage{tocloft}
\renewcommand{\cftdot}{.}
\renewcommand{\cftsecdotsep}{\cftdotsep}

% Пакет для библиографии
% Для компиляции с библиографией используй рецепт: xelatex -> biber -> xelatex x2
%\usepackage[backend=biber, style=gost-numeric, language=autobib, autolang=other]{biblatex}
%\addbibresource{references.bib}

% Пакет для создания графиков
%\usepackage{pgfplots}
%\pgfplotsset{compat=1.18}
%\pgfplotsset{/pgf/number format/use comma} % Использование запятой в качестве десятичного разделителя

% Пакеты для листинга кода и его подсветки
% \usepackage{listings}
% \usepackage{xcolor}

% Параметры документа (дисциплина, тема, преподаватель и т.п.)
% ==========================================
% ПАРАМЕТРЫ ДОКУМЕНТА
% ==========================================

% Библиография:
\newif\ifbib \bibfalse

% Графики (pgfplots):
\newif\ifplots \plotsfalse

% Листинг кода (listings + xcolor):
\newif\iflisting \listingfalse

% Кафедра (две строки)
\newcommand{\kafedraFirstString}{...}
\newcommand{\kafedraSecondString}{...}

% Дисциплина
\newcommand{\discipline}{...}

% Вид работы: «лабораторной» или «семинарской»
\newcommand{\workType}{...}

% Данные студента
\newcommand{\courseYear}{3}
\newcommand{\levelEducation}{бакалавриата}
\newcommand{\group}{ИДБ-23-14}
\newcommand{\lastName}{Дамакальщикова}
\newcommand{\firstName}{Михаила}
\newcommand{\middleName}{Александровича}

% Тема работы
\newcommand{\topic}{...}

% Направление и профиль подготовки
\newcommand{\specialization}{09.03.03 Прикладная информатика}
\newcommand{\trainingProfile}{Математическое и компьютерное моделирование процессов и систем}

% Дата сдачи
\newcommand{\dateSubmission}{}
\newcommand{\monthSubmission}{}
\newcommand{\yearSubmission}{}

% Преподаватель
\newcommand{\teacher}{...}

% Год в колонтитуле
\newcommand{\yearFooter}{...}


% Начало документа
\begin{document}

    % Отключает нумерацию страниц в титульном листе и в содержании
    \pagenumbering{gobble}

    % Титульная страница
    % Титульная страница
\begin{titlepage}

% Логотип вуза
\begin{figure}
    \centering
    \includegraphics[width=0.35\linewidth]{logo.png}
\end{figure}

% Подпись под логотипом
\begin{center}
    \textbf{МИНОБРНАУКИ РОССИИ} \\[2pt]
    \textbf{Федеральное государственное автономное образовательное учреждение} \\
    \textbf{высшего образования} \\
    \textbf{«Московский государственный технологический университет «СТАНКИН»} \\
    \textbf{(ФГАОУ ВО «МГТУ «СТАНКИН»)} \\[4pt]
    % Разделительная линия
    \hrule
\end{center}

% Институт
\noindent
\begin{minipage}[t]{0.5\textwidth}
    \raggedright
    \textbf{Институт} \\
    \textbf{информационных} \\
    \textbf{технологий}
\end{minipage}
% Кафедра
\hfill
\begin{minipage}[t]{0.5\textwidth}
    \raggedleft
    \textbf{Кафедра} \\
    \textbf{\kafedraFirstString} \\
    \textbf{\kafedraSecondString}
\end{minipage}
\vspace{1cm}

% Название работы
\begin{center}
    Отчёт о выполнении \workType~работы по дисциплине \\[2pt] «\discipline»
\end{center}
\vspace{0.5cm}
% Кто выполнил работу
\begin{center}
    Студента
    \hspace{0.2ex}
    \underline{\hspace{1ex}\textit{\courseYear}\hspace{1ex}}
    \hspace{0.2ex}
    курса
    \begin{tabular}[t]{c}
        \underline{\hspace{3ex}\textit{\levelEducation}\hspace{3ex}}
        \\[-4pt]
        {\tiny\textit{(уровень профессионального образования)}}
    \end{tabular}
    группы
    \hspace{0.2ex}
    \underline{\hspace{1ex}\textit{\group}\hspace{1ex}} \\[10pt]
    % ФИО
    \underline{\makebox[12cm]{\lastName~\firstName~\middleName}}
\end{center}
\vspace{0.5cm}
% Тема работы
\begin{center}
    На тему \\[3pt] \underline{\makebox[12cm]{\topic}}
\end{center}
\vspace{2cm}

% Направление и профиль подготовки
\noindent
\begin{minipage}[t]{0.5\textwidth}
    \raggedright
    Направление: \\[5pt]
    Профиль подготовки:
\end{minipage}
\begin{minipage}[t]{0.5\textwidth}
    \raggedleft
    \specialization \\[5pt]
    \trainingProfile
\end{minipage}
\vspace{1cm}

% Дата сдачи отчёта
\noindent
Отчёт сдан~~~«\underline{\makebox[4em]{\vphantom{\dateSubmission}}}»~~~\underline{\makebox[10em]{\vphantom{\monthSubmission}}}~~~20\underline{\makebox[2em]{\vphantom{\yearSubmission}}} г.

\vspace{0.5cm}

\noindent
% Оценка
Оценка:~~~\underline{\makebox[3cm]{\vphantom{text}}}

\vspace{0.5cm}

% Преподаватель
\noindent
\begin{minipage}[t]{0.17\textwidth}
    Преподаватель
\end{minipage}
\begin{minipage}[t]{0.7\textwidth}
    \centering
    \underline{\makebox[12cm]{\teacher}} \\
    {\small\textit{(Ф.И.О., должность, степень, звание.)}}
\end{minipage}
\hfill
\begin{minipage}[t]{0.1\textwidth}
    \centering
    \underline{\makebox[2cm]{\vphantom{\teacher}}} \\
    {\small\textit{(подпись)}}
\end{minipage}
\vfil

% Колонтитул
\enlargethispage{3cm}
\centering
\small Москва, \yearFooter~г.

\end{titlepage}


    % Содержание
    \begin{center}
        \tableofcontents
    \end{center}

    % Новая страница
    \newpage

    % Нумерация с 3-его номера
    \pagenumbering{arabic}
    \setcounter{page}{3}
	
    % ==========================================
    % СОДЕРЖАНИЕ ОТЧЁТА
    % ==========================================

    \section{Начало лабораторной работы}

        \unnsection{Раздел в главе №1}

    Какой-то текст.

    \unnsubsection{Подраздел в главе №1}

        Тоже какой-то текст.


    % ==========================================
    % БИБЛИОГРАФИЯ (раскомментируй при использовании)
    % ==========================================
    
    % Страница с библиографией
    %\cleardoublepage

    % Делаем так, чтобы библиография появлялась независимо от ссылок
    %\nocite{*}

    % Выводим библиографию по центру
    %\begin{center}
	    %\printbibliography[heading=bibintoc, title={Список литературы}]
    %\end{center}

    % Отключаем нумерацию страницы
    %\thispagestyle{empty}

    % Конец страницы с библиографией
    %\cleardoublepage

% Конец документа
\end{document}