% Указываем VS Code использовать XeLaTeX
% !TEX program = xelatex
\PassOptionsToPackage{pdfpagelabels=false}{hyperref}

% Шрифт всего документа - 12pt. Тип документа - статья
\documentclass[12pt]{article}

% Пакет для управления шрифтами
\usepackage{fontspec} 
% Шрифт по умолчанию
\setmainfont{Liberation Serif}
% \setmainfont{Times New Roman}  % для Windows/macOS
% \setmainfont{Liberation Serif} % для Linux

% Шрифт для листинга (раскомментируй при необходимости)
\setmonofont{DejaVu Sans Mono}[Scale=MatchLowercase]
% Альтернативы:
% \setmonofont{Courier New}      % для Windows/macOS
% \setmonofont{Liberation Mono}  % для Linux

% Пакет для работы с языками
\usepackage{polyglossia}
% Язык документа по умолчанию - русский
\setdefaultlanguage{russian}
% Дополнительный язык
\setotherlanguage{english}

% Пакет для использования запятой в качестве разделителя в дробных числах
\usepackage{icomma}

% Пакет для регулирования полей
\usepackage[a4paper, margin=2cm]{geometry}

% Пакет для вставки изображений
\usepackage{graphicx}
% Указываем путь к изображениям (удобно для организации проекта)
\graphicspath{{./images/}}

% Пакет для ссылок в документе
\usepackage{hyperref}

% Улучшения для оглавлений/закладок PDF: загружать после hyperref
\usepackage{bookmark}

% Пакет для разделительных символов в содержании
\usepackage{tocloft}
\renewcommand{\cftdot}{.}
\renewcommand{\cftsecdotsep}{\cftdotsep}

% Пакет для библиографии
% Для компиляции с библиографией используй рецепт: xelatex -> biber -> xelatex x2
%\usepackage[backend=biber, style=gost-numeric, language=autobib, autolang=other]{biblatex}
%\addbibresource{references.bib}

% Пакет для создания графиков
%\usepackage{pgfplots}
%\pgfplotsset{compat=1.18}

% Пакеты для листинга кода и его подсветки
% \usepackage{listings}
% \usepackage{xcolor}

% Параметры документа(дисциплина, год, тема и т.п.)
\newcommand{\kafedraFirstString}{...}
\newcommand{\kafedraSecondString}{...}
\newcommand{\discipline}{...}
\newcommand{\numberOfWork}{№... }
\newcommand{\topic}{...}
\newcommand{\teacher}{...}
\newcommand{\yearr}{202... г.}

% Начало документа
\begin{document}

    % Отключает нумерацию страниц в титульном листе и в содержании
    \pagenumbering{gobble}

    % Титульная страница
    \begin{titlepage}
	
    % Логотип вуза
    \begin{figure}
	       \centering
	       \includegraphics[width=0.4\linewidth]{logo.png}
    \end{figure}
	
    % Подпись под логотипом
    \begin{center}
	       \textbf{МИНОБРНАУКИ РОССИИ} \\
	       \textbf{Федеральное государственное автономное образовательное учреждение} \\
	       \textbf{высшего образования} \\
	       \textbf{«Московский государственный технологический университет «СТАНКИН»} \\
	       \textbf{(ФГАОУ ВО «МГТУ «СТАНКИН»)}
		
	       % Разделительная линия
	       \noindent \hrule
    \end{center}
	
    % Институт
    \noindent
    \begin{minipage}{0.5\textwidth}
	       \raggedright
	       \textbf{Институт} \\
	       \textbf{информационных} \\
	       \textbf{технологий}
    \end{minipage}
    % Кафедра
    \hfill
    \begin{minipage}{0.5\textwidth}
	       \raggedleft
	       \textbf{Кафедра} \\
	       \textbf{\kafedraFirstString} \\
	       \textbf{\kafedraSecondString}
    \end{minipage}
    \vspace{1cm}
	
    % Название работы
    \begin{center}
	       Лабораторная работа \numberOfWork  по дисциплине \\ «\discipline»
    \end{center}
    \vspace{1cm}
	
    % Тема работы
    \begin{center}
	       Тема \\ «\topic»
    \end{center}
    \vspace{3cm}
	
    % Выполнил
    \noindent
    \begin{minipage}{0.5\textwidth}
	       \raggedright
	       Выполнил студент группы ИДБ-23-14: \\
	       \centering
	       Дамакальщиков М. А.
    \end{minipage}
    % Преподаватель
    \hfill
    \begin{minipage}{0.5\textwidth}
	       \raggedleft
	       Проверил преподаватель: \\
	       \teacher
    \end{minipage}
    \vfill
	
    % Конец титульной страницы
    \centering
    \small Москва, \yearr
	
    \end{titlepage}

    % Содержание
    \begin{center}
        \tableofcontents
    \end{center}

    % Новая страница
    \newpage

    % Нумерация с 3-его номера
    \pagenumbering{arabic}
    \setcounter{page}{3}
	
    % ==========================================
    % ЗДЕСЬ ПИШИ СВОЙ КОНТЕНТ
    % ==========================================

    \section{Начало лабораторной работы}

        \unnsection{Раздел в главе №1}

    Какой-то текст.

    \unnsubsection{Подраздел в главе №1}

        Тоже какой-то текст.


    % ==========================================
    % БИБЛИОГРАФИЯ (раскомментируй при использовании)
    % ==========================================
    
    % Страница с библиографией
    %\cleardoublepage

    % Делаем так, чтобы библиография появлялась независимо от ссылок
    %\nocite{*}

    % Выводим библиографию по центру
    %\begin{center}
	       %\printbibliography[heading=bibintoc, title={Список литературы}]
    %\end{center}

    % Отключаем нумерацию страницы
    %\thispagestyle{empty}

    % Конец страницы с библиографией
    %\cleardoublepage

% Конец документа
\end{document}